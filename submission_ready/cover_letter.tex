\documentclass[12pt,a4paper]{article}
\usepackage{amsmath}
\usepackage{amssymb}
\usepackage{graphicx}
\usepackage{hyperref}
\usepackage{natbib}
\usepackage{url}
\usepackage{xcolor}
\usepackage{setspace}
\usepackage{geometry}
\usepackage{enumitem}
\usepackage{fancyhdr}
\usepackage{microtype}
\usepackage[T1]{fontenc}
\usepackage{lmodern}

% Set page geometry
\geometry{
 a4paper,
 total={170mm,257mm},
 left=25mm,
 right=25mm,
 top=25mm,
 bottom=25mm,
}

% Configure hyperref
\hypersetup{
    colorlinks=true,
    linkcolor=blue,
    filecolor=magenta,
    urlcolor=blue,
    citecolor=blue,
    pdftitle={Cover Letter - Enhancing Nitrate Removal in Denitrifying Woodchip Bioreactors},
    pdfauthor={Reza Moghaddam and Laura E. Christianson},
}

% Header and footer
\pagestyle{fancy}
\fancyhf{}
\fancyhead[L]{\footnotesize Cover Letter}
\fancyhead[R]{\footnotesize \today}
\fancyfoot[C]{\footnotesize Page \thepage}
\renewcommand{\headrulewidth}{0.4pt}
\renewcommand{\footrulewidth}{0.4pt}

\begin{document}

\begin{flushright}
\today
\end{flushright}

\vspace{1cm}

\noindent Dear Editor,

\vspace{0.5cm}

We are pleased to submit our revised manuscript entitled \textbf{``Enhancing Nitrate Removal in Denitrifying Woodchip Bioreactors: A Comprehensive Analysis of Enhancement Strategies and Environmental Trade-offs''} for consideration in your journal.

This comprehensive systematic review synthesizes current knowledge on enhancement strategies for woodchip bioreactors, addressing a critical need for improved nitrate removal technologies in agricultural and environmental water treatment applications. Our analysis of 70 peer-reviewed studies provides the first comprehensive evaluation of enhancement approaches including carbon supplementation, alternative media, bioaugmentation, and hybrid systems.

\section*{Key Contributions}

\begin{enumerate}[leftmargin=*]

\item \textbf{Quantitative Performance Assessment}: We provide the first standardized comparison of enhancement strategies, demonstrating that carbon dosing achieves 5.1--8.6 g N/m$^3$/day removal rates while alternative media approaches reach 12.8--15.2 g N/m$^3$/day. All economic data has been standardized to 2023 USD using Consumer Price Index adjustment factors for accurate cost comparisons.

\item \textbf{Environmental Trade-off Analysis}: Our comprehensive assessment addresses potential pollution swapping, including detailed analysis of dissolved organic carbon leaching, phosphorus dynamics, and greenhouse gas emissions. We provide practical mitigation strategies for minimizing environmental impacts while maximizing treatment performance.

\item \textbf{Temperature Sensitivity Framework}: We present the first systematic analysis of temperature effects across enhancement strategies, revealing substantial variability (Q$_{10}$ = 1.8--3.0) and introducing advanced modeling approaches including Macro-molecular Rate Theory (MMRT) as alternatives to traditional Arrhenius-based models.

\item \textbf{Economic Cost-Effectiveness Analysis}: Our standardized economic assessment demonstrates cost-effectiveness ranging from \$10.56--86/kg N removed, with comprehensive lifecycle considerations and comparative analysis against conventional agricultural treatment options.

\item \textbf{Practical Implementation Guidance}: We provide stakeholder-specific recommendations for researchers, funding agencies, practitioners, and policymakers, with clear emphasis on the developmental nature of these enhancement strategies and the need for site-specific validation.

\item \textbf{Comprehensive Mitigation Strategies}: Our analysis includes detailed operational and design approaches for minimizing greenhouse gas emissions, optimizing hydraulic retention time (8--16 hours), managing carbon dosing, and implementing proper system controls.

\end{enumerate}

\section*{Significance and Impact}

This work addresses several critical gaps in the current literature by providing standardized performance comparisons, comprehensive environmental impact assessment, and practical guidance for technology implementation. The manuscript presents both the potential benefits and limitations of enhancement strategies, offering a balanced perspective essential for informed decision-making in water quality management.

We believe this comprehensive analysis will be of significant interest to your readers, particularly those involved in agricultural water management, environmental engineering, and water quality protection. The systematic approach and practical recommendations make this work valuable for both researchers and practitioners in the field.

\section*{Quality Assurance}

We have carefully addressed all previous reviewer comments and have substantially enhanced the manuscript with additional analysis, improved clarity, and comprehensive coverage of environmental trade-offs. All data presented is verified against authentic literature sources, ensuring scientific accuracy and reliability.

We confirm that this manuscript has not been published elsewhere and is not under consideration for publication in another journal. All authors have approved the manuscript and agree to its submission.

\vspace{0.5cm}

Thank you for considering our manuscript for publication. We look forward to your response and are available to address any questions or provide additional information as needed.

\vspace{1cm}

\noindent Sincerely,

\vspace{1.5cm}

\noindent \textbf{Reza Moghaddam}\\
Earth Sciences, New Zealand\\
Email: \href{mailto:reza.moghaddam@niwa.co.nz}{reza.moghaddam@niwa.co.nz}

\vspace{1cm}

\noindent \textbf{Laura E. Christianson}\\
Research Associate Professor\\
Department of Crop Sciences\\
University of Illinois at Urbana-Champaign\\
S-322 Turner Hall, Urbana, IL 61801, USA

\end{document}