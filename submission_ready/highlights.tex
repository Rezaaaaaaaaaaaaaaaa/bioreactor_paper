\documentclass[12pt,a4paper]{article}
\usepackage{amsmath}
\usepackage{amssymb}
\usepackage{graphicx}
\usepackage{hyperref}
\usepackage{natbib}
\usepackage{url}
\usepackage{xcolor}
\usepackage{setspace}
\usepackage{geometry}
\usepackage{enumitem}
\usepackage{fancyhdr}
\usepackage{microtype}
\usepackage[T1]{fontenc}
\usepackage{lmodern}

% Set page geometry
\geometry{
 a4paper,
 total={170mm,257mm},
 left=25mm,
 right=25mm,
 top=25mm,
 bottom=25mm,
}

% Configure hyperref
\hypersetup{
    colorlinks=true,
    linkcolor=blue,
    filecolor=magenta,
    urlcolor=blue,
    citecolor=blue,
    pdftitle={Highlights - Enhancing Nitrate Removal in Denitrifying Woodchip Bioreactors},
    pdfauthor={Reza Moghaddam and Laura E. Christianson},
}

% Header and footer
\pagestyle{fancy}
\fancyhf{}
\fancyhead[L]{\footnotesize Highlights}
\fancyhead[R]{\footnotesize \today}
\fancyfoot[C]{\footnotesize Page \thepage}
\renewcommand{\headrulewidth}{0.4pt}
\renewcommand{\footrulewidth}{0.4pt}

\title{\textbf{Highlights}\\
\large Enhancing Nitrate Removal in Denitrifying Woodchip Bioreactors: A Comprehensive Analysis of Enhancement Strategies and Environmental Trade-offs}

\author{Reza Moghaddam\textsuperscript{1,*} and Laura E. Christianson\textsuperscript{2}}
\date{}

\begin{document}

\maketitle

\begin{center}
\footnotesize
\textsuperscript{1}Earth Sciences New Zealand\\
\textsuperscript{2}Research Associate Professor, Department of Crop Sciences, University of Illinois at Urbana-Champaign\\
S-322 Turner Hall, Urbana, IL 61801, USA\\
*Corresponding author: \href{mailto:reza.moghaddam@niwa.co.nz}{reza.moghaddam@niwa.co.nz}
\end{center}

\vspace{1cm}

\section*{Research Highlights}

\begin{itemize}[leftmargin=*, itemsep=0.8em]

\item \textbf{Comprehensive Systematic Review}: Systematic review of 70 studies evaluates enhancement strategies for woodchip bioreactors treating nitrate-contaminated water across multiple experimental scales and operational conditions.

\item \textbf{Quantified Performance Enhancement}: Carbon supplementation achieves 5.1--8.6 g N/m$^3$/day removal rates while alternative media approaches reach 12.8--15.2 g N/m$^3$/day, representing substantial improvements over conventional woodchip systems.

\item \textbf{Temperature Sensitivity Analysis}: Temperature sensitivity varies substantially among strategies (Q$_{10}$ = 1.8--3.0), with aged woodchips showing greater temperature dependence than fresh materials, providing critical design guidance for climate-specific applications.

\item \textbf{Economic Cost-Effectiveness}: Enhanced systems demonstrate cost-effectiveness ranging from \$10.56--86/kg N removed depending on strategy and operational conditions, with comprehensive lifecycle analysis and standardization to 2023 USD for accurate comparisons.

\item \textbf{Environmental Impact Mitigation}: Comprehensive mitigation strategies address environmental trade-offs including greenhouse gas emissions, dissolved organic carbon leaching, and phosphorus dynamics through optimized hydraulic retention time (8--16 hours) and proper system design.

\item \textbf{Comparative Economic Analysis}: Economic analysis reveals lifecycle considerations and comparative performance against conventional agricultural treatment options, including constructed wetlands (\$15--60/kg N) and precision nutrient management systems (\$5--25/kg N).

\end{itemize}

\vspace{1cm}

\section*{Practical Significance}

This work provides the first standardized framework for comparing woodchip bioreactor enhancement strategies, enabling informed decision-making for water quality management applications. The comprehensive analysis addresses critical knowledge gaps in environmental trade-off assessment while providing practical implementation guidance for diverse stakeholder groups.

\end{document}